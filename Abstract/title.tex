\thispagestyle{empty}

\vspace{0pt plus1fill} %число перед fill = кратность относительно некоторого расстояния fill, кусками которого заполнены пустые места
\begin{flushright}
  \large{На правах рукописи}\\\vskip5pt
  \includegraphics[height=2.6cm]{../images/my_signature} 
\end{flushright}
\vspace{-\baselineskip}
\vspace{-\baselineskip}

\vspace{0pt plus3fill} %число перед fill = кратность относительно некоторого расстояния fill, кусками которого заполнены пустые места
\begin{center}
\large\bf \thesisAuthor
\end{center}

\vspace{0pt plus3fill} %число перед fill = кратность относительно некоторого расстояния fill, кусками которого заполнены пустые места
\begin{center}
%\baselineskip=1.0cm
%\linespread{1.3}
\baselineskip=1.0cm
\textbf {\large \thesisTitle}

\vspace{0pt plus3fill} %число перед fill = кратность относительно некоторого расстояния fill, кусками которого заполнены пустые места
{\large \thesisSpecialtyNumber\ "---\par \thesisSpecialtyTitle}

\vspace{0pt plus1.5fill} %число перед fill = кратность относительно некоторого расстояния fill, кусками которого заполнены пустые места
\large{Автореферат}\par
\large{диссертации на соискание ученой степени\par \thesisDegree}
\end{center}

\vspace{0pt plus4fill} %число перед fill = кратность относительно некоторого расстояния fill, кусками которого заполнены пустые места
\begin{center}
{\large{\thesisCity\ "--- \thesisYear}}
\end{center}

\newpage
% оборотная сторона обложки
\thispagestyle{empty}
\noindent Работа выполнена \thesisOrganization.

\par\bigskip
%\begin{table}[h] % считается не очень правильным использовать окружение table, не задавая caption
    \noindent%
    \begin{tabular}{@{}lp{10.8cm}}
        \sfs Научный руководитель: & \sfs \supervisorFio \par
                                      \supervisorRegalia
        \vspace{4mm} \\
        {\sfs Официальные оппоненты:} &
        {\sfs \opponentOneFio\par
                  \opponentOneRegalia\par
                  \opponentOneJobPlace\par %\vspace{3mm}
                  \opponentOneJobPost\par \vskip3mm %\vspace{3mm}
                  \opponentTwoFio\par %\vspace{1mm}
                  \opponentTwoRegalia\par
                  \opponentTwoJobPlace\par
                  \opponentTwoJobPost\par
        }
        \vspace{4mm} \\
        {\sfs Ведущая организация:} & {\sfs \leadingOrganizationTitle }
    \end{tabular}  
%\end{table}
\par\bigskip

%\noindent Защита состоится \defenseDate~на~заседании диссертационного совета \defenseCouncilNumber~при \defenseCouncilTitle~по адресу: \defenseCouncilAddress.

\noindent Защита состоится 18 декабря 2018 года в 14 часов 30 минут на~заседании диссертационного совета \defenseCouncilNumber~при \defenseCouncilTitle~по адресу: \defenseCouncilAddress.

\vspace{5mm}
\noindent С диссертацией можно ознакомиться в библиотеке и на сайте НИУ МИЭТ: \\ \synopsisLibrary.

\vspace{5mm}
%\noindent{Автореферат разослан \synopsisDate.}

\noindent{Автореферат разослан~~~~~~~~~~~~~~~~~~~~~~~~~~~~~~~~~~~~~~2018 года.}

\vspace{5mm}
%\begin{table} [h] % считается не очень правильным использовать окружение table, не задавая caption
\par\bigskip
    \noindent %
    \begin{tabular}{@{}p{7.18cm} c r@{}}
        \begin{tabular}{@{}p{6.98cm} c r@{}}
            \sfs Ученый секретарь  \\
            \sfs диссертационного совета  \\
            \sfs \defenseSecretaryRegalia
        \end{tabular} 
    &
        \begin{tabular}{c}
        \includegraphics[height=2.6cm]{../images/gureev_signature} 
        \end{tabular} 
    &
        \begin{tabular}{r@{}}
            \\
            \\
            \sfs \hskip-33pt \defenseSecretaryFio
        \end{tabular} 
    \end{tabular}
%\end{table}
\newpage
