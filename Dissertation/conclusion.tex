\chapter*{Заключение}						% Заголовок
\addcontentsline{toc}{chapter}{Заключение}	% Добавляем его в оглавление

%% Согласно ГОСТ Р 7.0.11-2011:
%% 5.3.3 В заключении диссертации излагают итоги выполненного исследования, рекомендации, перспективы дальнейшей разработки темы.
%% 9.2.3 В заключении автореферата диссертации излагают итоги данного исследования, рекомендации и перспективы дальнейшей разработки темы.
%% Поэтому имеет смысл сделать эту часть общей и загрузить из одного файла в автореферат и в диссертацию:

Основные результаты диссертационной работы заключаются в следующем:
%% Согласно ГОСТ Р 7.0.11-2011:
%% 5.3.3 В заключении диссертации излагают итоги выполненного исследования, рекомендации, перспективы дальнейшей разработки темы.
%% 9.2.3 В заключении автореферата диссертации излагают итоги данного исследования, рекомендации и перспективы дальнейшей разработки темы.
% \begin{enumerate}
%   \item Сформулированы основные требования к системе управления имплантируемым роторным насосом крови: точная оценка расхода РНК на основе доступных параметров насоса, поддержание требуемого уровня расхода в различных физиологических условиях либо обеспечение физиологического потока через насос, соответствующего потребностям организма, и предотвращение неблагоприятных состояний в сердечно-сосудистой системе, обусловленных спецификой работы роторного насоса;
%   \item Разработан метод косвенной оценки потока через имплантируемый роторный насос крови с учетом инерционных и вязкостных свойств крови;  
%   \item Разработан метод определения режимов работы имплантируемого роторного насоса крови с целью управления неблагоприятными состояниями в сердечно-сосудистой системе; он позволяет идентифицировать следующие режимы работы: обратное течение крови через насос, частичная разгрузка желудочка с периодически открывающимся аортальным клапаном, полная разгрузка желудочка с постоянно закрытым аортальным клапаном, частичный и полный коллапс желудочка во время сердечного цикла;
%   \item На базе разработанных методов предложен алгоритм управления имплантируемым роторным насосом крови, который удовлетворяет основным требованиям к современной системе управления для аппаратов вспомогательного кровообращения, кроме обеспечения физиологического потока через насос;
%   \item Разработанный метод определения режимов работы имплантируемого роторного насоса крови успешно проверен с использованием результатов испытаний двух поколений РНК, используемых в первом российском коммерческом аппарате вспомогательного кровообращения <<Спутник>> на гидродинамическом стенде в динамических условиях.
% \end{enumerate}

\begin{enumerate}

    \item Разработана математическая модель сердечно-сосудистой системы, которая позволила исследовать взаимодействие имплантируемого роторного насоса крови и сердечно-сосудистой системы в условиях сердечной недостаточности. 
	\item Разработан алгоритм структурно-параметрической идентификации, который позволил построить математические модели имплантируемых роторных насосов крови на основе их расходно-напорных характеристик в соответствии с критериями оценки эффективности идентификации. 
	\item Проведено исследование взаимодействия имплантируемого роторного насоса крови и сердечно-сосудистой системы методами математического моделирования. \\В результате исследования разработаны метод определения режимов работы роторного насоса крови и способ управления роторным насосом крови, который позволяет поддерживать заданный уровень расхода насоса и предотвращать нежелательные режимы работы насоса, а также предложены следующие критерии, которые позволяют оценить эффективность идентификации для управления имплантируемым роторным насосом крови: точность оценки расхода насоса и точность определения перехода между режимами работы насоса. 
	\item Проведено исследование взаимодействия имплантируемого роторного насоса крови и сердечно-сосудистой системы с использованием экспериментальных данных для роторных насосов крови Спутник, полученных в испытательном гидродинамическом стенде. При этом для критериев оценки эффективности идентификации заданы следующие пороговые величины: средняя точность оценки расхода насоса не менее 90\% и точность определения переходов между режимами работы насоса не менее 80\%. \\В результате исследования с использованием алгоритма структурно-параметрической идентификации и в соответствии с критериями оценки эффективности идентификации построены математические модели имплантируемых роторных насосов крови, которые обеспечивают среднюю точность оценки расхода насоса не менее 90\% и точность определения переходов между режимами работы насоса более 91\%.

%идентификация исследованных насосов с использованием разработанного алгоритма и следующих пороговых величин для критериев оценки эффективности: средняя точность оценки расхода насоса не менее 90\% и точность определения переходов между режимами работы насоса не менее 85\%. Построенные математические модели, которые обеспечивают соответствие заданным пороговым величинам для критериев оценки эффективности со средней точностью оценки расхода насоса не менее 90\% и точностью определения переходов между режимами работы насоса более 91\%.  \\

%На основе проведенного исследования в алгоритм идентификации внесены изменения, которые позволяют осуществлять поиск математической модели согласно пороговым величинам для критериев оценки эффективности. Описанные изменения позволили построить математические модели двух насосов

%   \item Разработана математическая модель роторного насоса крови с целью косвенной оценки потока через насос, учитывающая инерционные и вязкостные эффекты крови.
%   \item Разработан метод определения режимов работы роторного насоса крови на основе его математической модели, который продемонстрировал точность не менее 80 \% при тестировании на модели сердечно-сосудистой системы.
%   \item Разработан алгоритм управления роторным насосом крови, который удовлетворяет основным требованиям по регулированию работы насоса для аппаратов вспомогательного кровообращения, используемых в клинической практике.
%   \item Предложен алгоритм разработки математической модели роторного насоса крови с использованием результатов испытаний роторных насосов в динамических условиях, который обеспечил среднюю точность оценки расхода насоса не менее 90 \%.
%   \item Разработанный метод определения режимов работы роторного насоса крови проверен с использованием результатов испытаний двух поколений роторных насосов на гидродинамическом стенде в двух состояниях, соответствующих различным степеням сердечной недостаточности, продемонстрировав точность не менее 90 \%.
\end{enumerate}

% Предложенный в данной работе метод оценки потока через роторный насос крови является косвенным, т.\:е. вычисляет поток на основе доступных параметров насоса с некоторой точностью. В данной работе в качестве входных параметров модели используются перепад давления в насосе, его скорости и величина вязкости крови. В реальных условиях после имплантации РНК постоянное неинвазивное отслеживание перепада давления не представляется возможным, поэтому переход к собственным параметрам насоса, таким как электрический ток двигателя, скорость вращения или противо-ЭДС, остается основной задачей на ближайшее будущее. Предполагается, что возможное использование датчика расхода позволит в реальном времени рассчитывать поправочный коэффициент в виде величины вязкости крови, что также будет являться дополнительным диагностическим параметром. 
% 
% Разработанный метод определения режимов работы роторного насоса крови позволяет определить неблагоприятные состояния в сердечно-сосудистой системе, связанные с обратным течением крови через насос или коллапсом желудочком сердца, включая закрытое состояние аортального клапана при работающем РНК, что в долгосрочной перспективе позволит сохранить его функциональность, и реализовать новые стратегии лечения в том числе направленные на восстановление миокарда. 
% 
% Предложенный метод также генерирует различные варианты сигналов, описывающих динамику течения крови через насос и изменяющихся согласованно с изменениями этой динамики, за счет множества производных, которые можно получить из математической модели РНК. Это позволяет расширить возможности алгоритмов, основанных на анализе временных диаграмм сигналов насоса.
% 
% Данный метод был успешно протестирован на математической модели сердечно-сосудистой система и на гидродинамическом стенде в динамических условиях. При этом установлено, что он универсален и может быть использован для любых существующих роторных насосов крови. Одно из возможных его применений заключается в использовании в качестве средства неинвазивной диагностики сердечно-сосудистой системы при наличии АВК. Тем не менее, необходимыми являются испытания на животных и результаты клинических наблюдений за пациентами.
% 
% %В данной работе было показано, что такой подход обладает универсальностью и может быть использован для любых существующих роторных насосов крови. Он также реализует возможность использования роторного насоса в качестве средства диагностики и представляет собой основу для систем адаптивного управления роторными насосами крови в рамках существующей технологии. 
% 
% Предложенный алгоритм управления роторным насосом крови не реализует возможность физиологического управления роторным насос, т.\:е. не позволяет обеспечить расход насоса, соответствующий физиологическим потребностям организма. Но справедливости ради стоит отметить, что ни один из доступных в клинической практике АВК также не обладает такой возможностью. Реализация такой опции в коммерческой системе управления остается делом будущего.
% 
% Также следует отметить, что необходимым вариантом тестирования алгоритма управления, который не был рассмотрен в данной работе, является состояние сердечно-сосудистой системы под физической нагрузкой.
% 
% % в рамках выбранного применения АВК, которая будет заключаться в поддержании определенного режима работы РНК и, соответственно, обеспечении определенного физиологического состояния сердечно-сосудистой системы. 
% % 
% % Мы также осознаем, что в реальных динамических условиях после имплантации насоса постоянное отслеживание давления не представляется возможным, поэтому рассчитываем перейти к собственным параметрам насоса, таким как электрический ток, скорость вращения или противо-ЭДС.


% Предложенный в данной работе метод оценки потока через роторный насос крови является косвенным, т.\:е. вычисляет поток на основе доступных параметров насоса с некоторой точностью. В данной работе в качестве входных параметров модели используются перепад давления в насосе, его скорости и величина вязкости крови. В реальных условиях после имплантации РНК постоянное неинвазивное отслеживание перепада давления не представляется возможным, поэтому переход к собственным параметрам насоса, таким как электрический ток двигателя, скорость вращения или противо-ЭДС, остается основной задачей на ближайшее будущее. Предполагается, что возможное использование датчика расхода позволит в реальном времени рассчитывать поправочный коэффициент в виде величины вязкости крови, что также будет являться дополнительным диагностическим параметром. 
% 
% Разработанный метод определения режимов работы роторного насоса крови позволяет определить неблагоприятные состояния в сердечно-сосудистой системе, связанные с обратным течением крови через насос или коллапсом желудочком сердца, включая закрытое состояние аортального клапана при работающем РНК, что в долгосрочной перспективе позволит сохранить его функциональность, и реализовать новые стратегии лечения в том числе направленные на восстановление миокарда. 
% 
% Предложенный метод также генерирует различные варианты сигналов, описывающих динамику течения крови через насос и изменяющихся согласованно с изменениями этой динамики, за счет множества производных, которые можно получить из математической модели РНК. Это позволяет расширить возможности алгоритмов, основанных на анализе временных диаграмм сигналов насоса.
% 
% Данный метод был успешно протестирован на математической модели сердечно-сосудистой система и на гидродинамическом стенде в динамических условиях. При этом установлено, что он универсален и может быть использован для любых существующих роторных насосов крови. Одно из возможных его применений заключается в использовании в качестве средства неинвазивной диагностики сердечно-сосудистой системы при наличии АВК. Тем не менее, необходимыми являются испытания на животных и результаты клинических наблюдений за пациентами.
% 
% %В данной работе было показано, что такой подход обладает универсальностью и может быть использован для любых существующих роторных насосов крови. Он также реализует возможность использования роторного насоса в качестве средства диагностики и представляет собой основу для систем адаптивного управления роторными насосами крови в рамках существующей технологии. 
% 
% Предложенный алгоритм управления роторным насосом крови не реализует возможность физиологического управления роторным насос, т.\:е. не позволяет обеспечить расход насоса, соответствующий физиологическим потребностям организма. Но справедливости ради стоит отметить, что ни один из доступных в клинической практике АВК также не обладает такой возможностью. Реализация такой опции в коммерческой системе управления остается делом будущего.
% 
% Также следует отметить, что необходимым вариантом тестирования алгоритма управления, который не был рассмотрен в данной работе, является состояние сердечно-сосудистой системы под физической нагрузкой.

% \clearpage
% \chapter*{Список сокращений и условных обозначений}						% Заголовок
% \addcontentsline{toc}{chapter}{Список сокращений и условных обозначений}
% 
% АВК -- Аппарат вспомогательного кровообращения
% 
% РНК -- Роторный насос крови
% 
% ССС -- Сердечно-сосудистая система
% 
% СН -- Сердечная недостаточность

% \clearpage
% \chapter*{Благодарности}
% \addcontentsline{toc}{chapter}{Благодарности}
% 
% Я хочу выразить свою признательность и благодарность:
% 
% \begin{itemize}
%  \item моему научному руководителю -- Селищеву Сергею Васильевичу -- за помощь в выборе направления исследования, за бесценные отзывы на диссертационную работы, за возможность участия в зарубежных конференциях и бесконечное терпение, 
%  \item руководителю лаборатории медицинской техники -- Телышеву Дмитрию Викторовичу -- за участие в обсуждении результатов диссертационной работы и бесценные отзывы, за помощь в написании научных работ и содействие в проведении экспериментальных исследований, за проявленное неравнодушие, % за понимание и отзывчивость, % внимание благожелательное отношение,  %в получении экспериментальных данных,
%  \item сотрудникам кафедры биомедицинских систем -- Нестеренко Игорю, Миндубаева Эдуарду, Гуськову Алексею и Денисову Максиму -- за участие в обсуждении результатов и помощь в оформлении диссертационной работы,
%  \item сотрудникам кафедры медицинских информационных технологий в институте Гельмгольца по биомедицинской инженерии (г. Ахен, Германия) -- в особенности Marian Walter и Daniel R{\"u}eschen -- за участие в обсуждение результатов диссертационной работы и возможность проведения экспериментальных исследований, % the Chair for Medical Information Technology (MedIT), Helmholtz-Institute for Biomedical Engineering (RWTH Aachen University)
%  \item моим родителям -- Петуховой Алле Петровне и Петухову Сергею Анатольевичу -- за моральную поддержку все эти годы.
% \end{itemize}
